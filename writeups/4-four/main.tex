\title{\textrm{deSQLifier (4)}}
\author{
        Kia Rahmani \\
                Department of Computer Science\\
        Purdue University, USA
}
\date{\today}
% Main Document
\documentclass[12pt,letter]{article}
\usepackage{xcolor,listings}
\usepackage{graphicx}
\usepackage[inline]{enumitem}
\usepackage[utf8]{inputenc}
\usepackage[english,ngerman]{babel}
\usepackage{amsmath}
\usepackage{mathtools}
\usepackage{mathpartir}
\usepackage{listings}
\usepackage{amsfonts}
\usepackage{color}

\definecolor{codegreen}{rgb}{0,0.6,0}
\definecolor{codegray}{rgb}{0.5,0.5,0.5}
\definecolor{codepurple}{rgb}{0.58,0,0.82}
\definecolor{backcolour}{RGB}{255, 219, 219}
\definecolor{backcolourSimp}{RGB}{219, 232, 255}

\lstdefinestyle{simpSQLstyle}{
    backgroundcolor=\color{backcolourSimp},   
    commentstyle=\color{codegreen},
    keywordstyle=\color{magenta},
    numberstyle=\tiny\color{codegray},
    stringstyle=\color{codepurple},
    basicstyle=\footnotesize,
    breakatwhitespace=false,         
    breaklines=true,                 
    captionpos=b,                    
    keepspaces=true,                 
    numbers=left,                    
    numbersep=5pt,                  
    showspaces=false,                
    showstringspaces=false,
    showtabs=false,                  
    tabsize=2
}
\lstdefinestyle{kvSQLstyle}{
    backgroundcolor=\color{backcolour},   
    commentstyle=\color{codegreen},
    keywordstyle=\color{magenta},
    numberstyle=\tiny\color{codegray},
    stringstyle=\color{codepurple},
    basicstyle=\footnotesize,
    breakatwhitespace=false,         
    breaklines=true,                 
    captionpos=b,                    
    keepspaces=true,                 
    numbers=left,                    
    numbersep=5pt,                  
    showspaces=false,                
    showstringspaces=false,
    showtabs=false,                  
    tabsize=2
}

\begin{document}
\selectlanguage{english}
\maketitle
%\begin{abstract} \end{abstract}

% ---------------------------------------------------
% ---------------------------------------------------
% Sections ------------------------------------------
% ---------------------------------------------------
% ---------------------------------------------------
\section{Syntax of simpSQL}
The following is the formal definition of the simpSQL language based on Kartik's document,
representing a simple programing language with realistic standard SQL queries.
% the syntax of standard SQL
\begin{figure}[h]
        $$
        tn \in \texttt{TableName} \qquad 
        f_{id},f_v \in \texttt{FieldName} \qquad 
        v \in \texttt{Variable} \qquad
        $$
        %
        \vspace{-6mm} %visually better
        %
        $$ \odot \in \{<,\leq,=,>,\geq\} \qquad 
        \oplus \in \{+,-,\times,/ \} \qquad 
        \otimes \in \{\wedge, \vee\}
        $$
        %
        $$
        \begin{matrix*}[l]
                
                

                t & ::= & (tn,\overline{f_{id}},\overline{f_{v}})\\
                e_d & ::= & f \;|\; v \;|\; e_d \oplus e_d \;|\; \mathbb{Z}\\ 
                \phi_d & ::= & f\odot e_d \;|\; f \; \texttt{IN} \; v \;|\; \neg \phi_d \;|\;\phi_d \otimes \phi_d  \\
                e_c & ::= &  v \;|\; \texttt{CHOOSE} \; v \;|\; e_c\oplus e_c \;|\; \texttt{NULL} \;|\; \mathbb{Z} \\
                \phi_c & ::= & v\odot e_c \;|\; \neg \phi_c \;|\;\phi_c \otimes \phi_c  \\
                \phi_j & ::= & f=f \;|\; \phi_j \wedge \phi_j  \\
                c   & ::=  & \texttt{SELECT}\; \bar{f} \;\texttt{AS}\; v \;\texttt{WHERE}\; \phi_d \;|\; 
		\texttt{SELECT}_{\phi_j}\; \bar{f} \;\texttt{AS}\; v\;\texttt{WHERE}\; \phi_d \;|\; \\
		& & \texttt{UPDATE SET}\; \bar{f}=\bar{e_c} \;\texttt{WHERE}\; \phi_d \;|\; \ v=e_c \;|\; c;c \;|\;\\ 
                & & \texttt{INSERT VALUES}\; \bar{f}=\bar{e_c} \;|\;
		\;\texttt{DELETE}\; tn \;\texttt{WHERE}\; \phi_d \;|\;\\
                & & \texttt{IF}\; \phi_c \;\texttt{THEN}\; c \;\texttt{ELSE}\; c \;|\; 
                \;\texttt{FOREACH}\; v_1 \;\texttt{in}\; v_2 \;\texttt{DO}\; c \;\texttt{END}\;
        \end{matrix*}
        $$
        \hrule \hrule

\caption{Syntax of simpSQL}
\label{fig:standard}
\end{figure}



% ---------------------------------------------------
\section{Syntax of kvSQL}
Figure \ref{fig:syn} presents the kvSQL language which is used to write
generic key-value store backed applications. The language is not very
different from SQL; however it replaces tables with (denormalized)
objects supporting restricted queries. This model represents the real world restrictions in EC stores such as Cassandra. We will later formally define the
translation from simpSQL to kvSQL
\begin{figure}[h]
        $$
        t \in \texttt{TableName} \qquad 
        f_{id},f_v \in \texttt{FieldName} \qquad 
        v_{val},v_{rec} \in \texttt{Variable} \qquad
        txn \in \texttt{TxnName}
        $$
        %
        \vspace{-6mm} %visually better
        %
        $$ \odot \in \{<,\leq,=,>,\geq\} \qquad 
        \oplus \in \{+,-,\times,/ \} \qquad 
        \otimes \in \{\wedge, \vee\}
        $$ 
        %
        $$
        \begin{matrix*}[l]
                pk & ::= & (\overline{f_{id}},f_v)\;|\;(\overline{f_{id}},f_{id})\\
                obj &  ::= & (t,pk,\overline{f_v}) \\
                r &  ::= & \bar{e}\;|\; \texttt{CHOOSE}(v_{rec}) \\
                e &  ::= & \mathbb{Z} \;|\; \texttt{NULL} \;|\; r^i \;|\; v_{val} \;|\; e \oplus e \\
                \phi_{pk}  & ::= & pk^{1} = \bar{e} \;|\; pk^{2} \odot e
                \;|\; \phi_{pk} \otimes \phi_{pk} \\
                op   & ::= & obj.\texttt{PUT}(r) \;|\; 
                v_{rec} = obj.\texttt{GET}(\phi_{obj^{2}}) \;|\;  obj.\texttt{DELETE}(\phi_{obj^{2}})  \\
                \phi_{c}  & ::= & e\odot e \;|\;  r\;\texttt{IN} \; v_{rec} \;|\; \phi_{c} 
                \otimes \phi_{c} \;|\; \neg \phi_{c} \\
                c   & ::=  & \{\overline{op}\}_{DC}
                \;|\; v_{val} = e \;|\; v_{rec} = \texttt{FILTER} (v_{rec})
                \\  & & \texttt{IF}\; \phi_c \;\texttt{THEN} \;c \;\texttt{ELSE}\; c \;|\;  c;c \;|\;
                \{c\}_{_{SER}} \;|\;
                \\  & & \texttt{FOREACH}\; v \;\texttt{IN} \; v \; \texttt{DO}\; c \;\texttt{END}
                
        \end{matrix*}
        $$
        \hrule \hrule
\caption{Syntax of kvSQL}
\label{fig:syn}
\end{figure}



% ---------------------------------------------------
\section{simpSQL to kvSQL Translation}
In this section we will present the complete algorithm to translate arbitrary simpSQL 
programs to an equivalent kvSQL verison. 
As an example, we will also apply our procedure on full TPC-C benchmark and
derive the kvSQL TPC-C.

\subsection{Data Remodeling Rules}
\textcolor{red}{//TODO}
\subsection{Data Remodeling: TPC-C}
% WAREHOUSE
\subsubsection*{SimpSQL Table: Warehouse}  
\begin{tabular}{ |c|c|c|c|c| }
 \hline
 \underline{w\_id} & w\_name & w\_address & w\_tax & w\_ytd \\
 \hline
 &   &   & & \\
 \hline
\end{tabular}

\subsubsection*{kvSQL Object(s): Warehouse}  
id := (w\_id) \\
warehouse\_by\_id :=
(Warehouse,(id,\_),[w\_name;w\_address;w\_tax;w\_ytd]) 
\\ \\
\hrule

%DISTRICT 
\subsubsection*{SimpSQL Table: District}  
\begin{tabular}{ |c|c|c|c|c|c| }
 \hline
 \underline{d\_id} & \underline{d\_w\_id} & d\_info & d\_ytd & d\_tax & d\_next\_o\_id\\
 \hline
 &   &   & & &\\
 \hline
\end{tabular}

\subsubsection*{kvSQL Object(s): District}  
 id := (d\_id,d\_w\_id) \\
 d\_info\_by\_id := 
(District,(id,\_),[d\_info])  \\
 d\_ytd\_by\_id := 
(District,(id,\_),[d\_ytd])  \\
 d\_tax\_by\_id := 
(District,(id,\_),[d\_tax])  \\
 d\_next\_o\_id\_by\_id := 
(District,(id,\_),[d\_next\_o\_id])  \\
\\ 
\hrule

%CUSTOMER
\subsubsection*{SimpSQL Table: Customer}  
\begin{tabular}{ |c|c|c|c|c|c|c|c| }
 \hline
 \underline{c\_id} & \underline{c\_d\_id} & \underline{c\_w\_id} &
 c\_name & c\_ytd &
 c\_delivery\_cnt & c\_payment\_cnt & c\_balance\\
 \hline
 &   &   & & & & &\\
 \hline
\end{tabular}

\subsubsection*{kvSQL Object(s): Customer}  
id := (c\_id,c\_d\_id,c\_w\_id) \\
 c\_name+ytd+...\_by\_id := 
(Customer,(id,\_),[c\_name;c\_ytd;...])  \\
 c\_balance\_by\_id := 
(Customer,(id,\_),[c\_balance])  \\
 c\_ytd+...\_by\_name := 
(Customer,(id,c\_name),[c\_ytd;...])  \\
 c\_balance\_by\_name := 
(Customer,(id,c\_name),[c\_balance])  \\
\\
\hrule

%ORDERS
\subsubsection*{SimpSQL Table: Orders}  
\begin{tabular}{ |c|c|c|c|c|c| }
 \hline
 \underline{o\_id} & \underline{o\_d\_id} & \underline{o\_w\_id} &
 o\_c\_id & o\_carrier\_id & o\_entry\_d\\
 \hline
 &   &   & & &\\
 \hline
\end{tabular}

\subsubsection*{kvSQL Object(s): Orders}  
id := (o\_id,o\_d\_id,o\_w\_id) \\
order\_by\_id := 
(Orders,(id,\_),[o\_c\_id;o\_carrier\_id;o\_entry\_d])  \\
o\_id+entryD+CarriedID\_by\_o\_c\_id := 
(Orders,(id,o\_c\_id),[o\_id;...])  \\
\\ 
\hrule


%ITEM
\subsubsection*{SimpSQL Table: Item}  
\begin{tabular}{ |c|c| }
 \hline
 \underline{i\_id} & i\_info\\
 \hline
 &   \\
 \hline
\end{tabular}

\subsubsection*{kvSQL Object(s): Item}  
id := (i\_id)\\
 i\_info\_by\_id := 
(Item,(id,\_),[i\_info]) \\
\\
\hrule


% ORDERLINE
\subsubsection*{SimpSQL Table: OrderLine}  
\begin{tabular}{ |c|c|c|c|c| }
 \hline
 \underline{ol\_o\_id} & \underline{ol\_d\_id} & \underline{ol\_w\_id} &
 \underline{ol\_number} & ol\_info\\
 \hline
 &   &   & & \\
 \hline
\end{tabular}

\subsubsection*{kvSQL Object(s): OrderLine}  
id := (ol\_o\_id,ol\_d\_id,ol\_w\_id,ol\_number) \\
ol\_info\_by\_id :=
(OrderLine,(id,\_),[ol\_info]) \\
ol\_number+info\_by\_ol\_o\_id :=
(OrderLine,(id,ol\_o\_id),[ol\_number;ol\_info]) \\
\\ 
\hrule


% STOCK
\subsubsection*{SimpSQL Table: Stock}  
\begin{tabular}{ |c|c|c|c|c| }
 \hline
 \underline{s\_i\_id} & \underline{s\_w\_id} & s\_quant & s\_order\_cnt &
 s\_info \\
 \hline
 &   &   & & \\
 \hline
\end{tabular}

\subsubsection*{kvSQL Object(s): Stock}  
id := (s\_i\_id,s\_w\_id) \\
s\_quant\_by\_id :=
(Stock,(id,\_),[s\_quant]) \\
s\_orderCnt\_by\_id :=
(Stock,(id,\_),[s\_order\_cnt]) \\
s\_info\_by\_id :=
(Stock,(id,\_),[s\_info]) \\
\\ 
\hrule


% ORDERLINE |><| STOCK 
\subsubsection*{SimpSQL Table: OrderLine JOIN Stock}
\begin{tabular}{ |c|c|c|c|c|c|c|c| }
 \hline
 \underline{ol\_o\_id} & \underline{ol\_d\_id} & \underline{ol\_w\_id} &
 \underline{ol\_number} & ol\_info &s\_i\_id & s\_w\_id & s\_quant \\
 \hline
 &   &   & & & & &\\
 \hline
\end{tabular}

\subsubsection*{kvSQL Object(s): OrderLine JOIN Stock}  
id := (ol\_o\_id,ol\_d\_id,ol\_w\_id,ol\_number) \\
s\_quant\_by\_ol\_o\_id :=
(OrderLine $\bowtie$ Stock,(id,ol\_o\_id),[s\_quant]) \\
ol\_by\_s\_i\_id :=
(OrderLine $\bowtie$ Stock,(id,s\_i\_id),[ol\_o\_id,...]) \\
\\ 
\hrule


% NEWORDER
\subsubsection*{SimpSQL Table: NewOrder}
\begin{tabular}{ |c|c|c| }
 \hline
 \underline{ol\_o\_id} & \underline{ol\_d\_id} & \underline{ol\_w\_id}\\
 \hline
 &   &  \\
 \hline
\end{tabular}

\subsubsection*{kvSQL Object(s): NewOrder}  
id := (no\_o\_id,no\_d\_id,no\_w\_id) \\
no\_by\_no\_d\_id :=
(NewOrder,(id,no\_d\_id),[]) \\
\\ 
\hrule

%History
\subsubsection*{SimpSQL Table: History}  
\begin{tabular}{ |c|c| }
 \hline
 \underline{h\_id} & h\_info\\
 \hline
 &   \\
 \hline
\end{tabular}

\subsubsection*{kvSQL Object(s): History}  
id := (h\_id)\\
 h\_info\_by\_id := 
(Item,(id,\_),[h\_info]) \\
\\
\hrule




\vspace{10mm}
\subsection{Program Rewriting Rules}
\textcolor{red}{//TODO}

\subsection{Program Rewriting: TPC-C}
%%%%%%%%%%%%%%%%%%%%%%%%%%%%%%%%%%%%%%%%%%%%%%%%%%%%%%%%%%%%%%%%%%%%%%%%%%%%%%%%%%%%%%%%%%%%%%%%%%%%%%5
%%%%%%%% NEW ORDER
\paragraph{New Order}:
\lstset{style=simpSQLstyle}
%% SIMP
\begin{lstlisting}[language=Python, caption=simpSQL,escapeinside={(*}{*)}]
NewOrder((*wh\_id,dist\_id,cust\_id,item\_list*)) :=
	(*$ \;\texttt{SELECT}\; w\_tax \;\texttt{AS}\; wx \;\texttt{WHERE}\; w\_id = wh\_id $*)
	(*$ \;\texttt{SELECT}\; (d\_tax,d\_next\_o\_id) \;\texttt{AS}\; dtx \;\texttt{WHERE}\; d\_id=dist\_id \wedge  d\_w\_id = wh\_id$*)
	(*$ \;\texttt{UPDATE SET}\; d\_next\_o\_id=dtx^{2}+1 \;\texttt{WHERE}\; d\_id=dist\_id \wedge  d\_w\_id = wh\_id$*)
	(*$ \;\texttt{SELECT}\; (c\_discount,...) \;\texttt{AS}\; cx \;\texttt{WHERE}\; c\_id = cust\_id \wedge c\_d\_id=dist\_id \wedge ...$*)
	(*$ \;\texttt{INSERT VALUES}\; (o\_id,o\_c\_id,...)=(dtx^2,cust\_id,...)$*)
	(*$ \;\texttt{INSERT VALUES}\; (no\_o\_id,no\_d\_id,no\_w\_id)=(dtx^2,dist\_id,wh\_id)$*)
	(*$ \;\texttt{FOREACH}\; i \;\texttt{IN}\; item\_list \;\texttt{DO}\;  $*)
		(*$ \;\texttt{SELECT}\; i\_info \;\texttt{AS}\; ix \;\texttt{WHERE}\; i\_id = i $*)
		(*$ \;\texttt{SELECT}\; (s\_quant,s\_orderCnt,...) \;\texttt{AS}\; sx \;\texttt{WHERE}\; s\_i\_id = i \wedge s\_w\_id = wh\_id $*)
		(*$ \;\texttt{IF}\; sx^1 - i^{quant} < 10 $*)
		(*$ \;\texttt{THEN}\; $*) (*$sqx = sx^1 - i^{quant} + 91$*)
		(*$ \;\texttt{ELSE}\; $*) (*$sqx = sx^1 - i^{quant}$*)
		(*$ \;\texttt{UPDATE SET}\; (s\_orderCnt,s\_quant,...)=(sx^2+1,sqx,...) \;\texttt{WHERE}\; s\_i\_id = i \wedge ...$*)
		(*$ \;\texttt{INSERT VALUES}\; (...,ol\_number,...)=(...,unique,...)$*)
	(*$ \;\texttt{END}$*)
	

	
\end{lstlisting}
%%%%

\lstset{style=kvSQLstyle}
\begin{lstlisting}[language=Python, caption=kvSQL,escapeinside={(*}{*)}]
# some non-interesting updates are eliminated
NewOrder((*wh\_id,dist\_id,cust\_id,item\_list,ol\_quant*)) := {	
	wx(*$= (warehouse\_by\_id).\texttt{GET}$ (id=wh\_id)*) #Retrieve warehouse by PK
	dtx(*$= (d\_tax\_by\_id).\texttt{GET}$ (id=dist\_id)*) #Retrieve d_tax by PK
	#Update d_next_o_id by PK:
	dnoix(*$= (d\_next\_o\_id\_by\_id).\texttt{GET}$ (id=dist\_id)*) 
	(*$ (d\_next\_o\_id\_by\_id).\texttt{PUT}$(dnoix[d\_next\_o\_id $\mapsto$d\_next\_o\_id+1])*);
	cx(*$= (c\_info\_by\_id).\texttt{GET}$ (id=(cust\_id,...))*) #Retrieve customer by PK
	#Enter new rows into Order and NewOrder objects (3 Objects):
	(*$ (order\_by\_id).\texttt{PUT}$(...)*); #new row is created from known values
	(*$ (o\_info\_by\_o\_c\_id).\texttt{PUT}$(...)*); #structure of the new row should match the denormalized object
	(*$ (no\_by\_d\_id).\texttt{PUT}$(...)*);
	
	FOREACH item_id IN (*item\_list*) DO
		ix(*$= (item\_info\_by\_id).\texttt{GET}$ (id=item\_id)*)
		#Retrieve Stock information by PK (from 3 objects):
		socx(*$= (s\_orderCnt\_by\_id).\texttt{GET}$ (id=(item\_id,...))*)
		sqx(*$= (s\_quant\_by\_id).\texttt{GET}$ (id=(item\_id,...))*)
		six(*$= (s\_info\_by\_id).\texttt{GET}$ (id=(item\_id,...))*)
		IF ((*sqx - ol\_quant $<$ 10*))
			(*$
			(s\_quant\_by\_id).\texttt{PUT}$(sqx[s\_quant$\mapsto$(s\_quant-ol\_quant+91)])*);		
			olx(*$= (ol\_by\_s\_id).\texttt{GET}$ (s\_id=(item\_id))*) #All OL using this stock
			FOREACH o_id IN (*ol\_x*) DO
				(*$ (s\_quant\_by\_ol\_o\_id).\texttt{PUT}$(...,sqx[s\_quant $\mapsto$(s\_quant - ol\_quant+91)],...)*);
			END;
		ELSE
			(*$ (s\_quant\_by\_id).\texttt{PUT}$(sqx[s\_quant $\mapsto$s\_quant - ol\_quant])*);		
			olx(*$= (ol\_by\_s\_id).\texttt{GET}$ (s\_id=(item\_id))*) #All OL using this stock
			FOREACH o_id IN (*ol\_x*) DO #update the denormalized join object
				(*$ (s\_quant\_by\_ol\_o\_id).\texttt{PUT}$(...,sqx[s\_quant $\mapsto$(s\_quant - ol\_quant)],...)*);
			END;
	
		#Enter a new order line (4 objects):
		(*$ (ol\_info\_by\_id).\texttt{PUT}$(...)*); #insert a new row from known values
		(*$ (ol\_number+info\_by\_ol\_o\_id).\texttt{PUT}$(...)*);#same values; dnrmlz'd object
		(*$ (s\_quant\_by\_ol\_o\_id).\texttt{PUT}$(...)*); #known values; insert in join object
		(*$ (ol\_by\_s\_id).\texttt{PUT}$(...)*); #insert in the denormalized join object
	END;

}(*$_{SER}$*)
\end{lstlisting}
\vspace{14mm}



\hrule
\hrule
\hrule
\hrule
\hrule
%%%%%%%%%%%%%%%%%%%%%%%%%%%%%%%%%%%%%%%%%%%%%%%%%%%%%%%%%%%%%%%%%%%%%%%%%%%%%%%%%%%%%%%%%%%%%%%%%%%%%%5
%%%%%%%%%%%%%%%%%%%%%%%%%%%%%%%%%%%%%%%%%%%%%%%%%%%%%%%%%%%%%%%%%%%%%%%%%%%%%%%%%%%%%%%%%%%%%%%%%%%%%%5
\vspace{2mm}


\paragraph{Payment}:
%--------- Payment
\lstset{style=simpSQLstyle}
\begin{lstlisting}[language=Python, caption=simpSQL,escapeinside={(*}{*)}]
Payment((*wh\_id,dist\_id,cust\_id,cust\_name,amnt*)) :=
	(*$ \;\texttt{SELECT}\; (w\_ytd) \;\texttt{AS}\; wx \;\texttt{WHERE}\; w\_id = wh\_id$*)
	(*$ \;\texttt{UPDATE SET}\; w\_ytd=wx^1+1 \;\texttt{WHERE}\; w\_id = wh\_id$*)
	(*$ \;\texttt{SELECT}\; (d\_ytd) \;\texttt{AS}\; dx \;\texttt{WHERE}\; d\_id = dist\_id \wedge w\_id = wh\_id$ *)
	(*$ \;\texttt{UPDATE SET}\; d\_ytd=dx^1+1 \;\texttt{WHERE}\; d\_id = dist\_id \wedge w\_id = wh\_id$ *)
	(*$ \;\texttt{IF}\; cust\_id = \texttt{NULL} $*) (*$ \;\texttt{THEN}\; $*) 
		(*$\texttt{SELECT}\; (c\_id,c\_balance,c\_ytd\_payment...) \;\texttt{AS}\; cx1 \;\texttt{WHERE}\; c\_name = cust\_name \wedge ...$*)
		(*$cx = \texttt{CHOOSE}\; cx1 $*) 
		(*$\texttt{UPDATE SET}\; (c\_balance,c\_ytd\_payment,...)=(cx^2-amnt,cx^3+amnt,...) $*)
			(*$\texttt{WHERE}\; c\_id = cx^1)$ *)
	(*$ \;\texttt{ELSE}\; $*)
		(*$\texttt{SELECT}\; (c\_balance,c\_ytd\_payment...) \;\texttt{AS}\; cx \;\texttt{WHERE}\; c\_id = cust\_id \wedge ...$*)
		(*$\texttt{UPDATE SET}\; (c\_balance,c\_ytd\_payment,...)=(cx^2-amnt,cx^3+amnt,...) $*)
			(*$\texttt{WHERE}\; c\_id = cust\_id)$ *)
	(*$ \;\texttt{INSERT VALUES}\; (h\_id,h\_info)=(unique,...)$*)
\end{lstlisting}
%%%%%

\lstset{style=kvSQLstyle}
%--------- Payment
\begin{lstlisting}[language=Python, caption=kvSQL,escapeinside={(*}{*)}]
Payment((*wh\_id,dist\_id,cust\_id,cust\_name,amnt*)) := {
	wx(*$= (warehouse\_by\_id).\texttt{GET}$ (id=wh\_id)*) #Retrieve warehouse by PK
	(*$ (warehouse\_by\_id).\texttt{PUT}$(wx[w\_ytd $\mapsto$w\_ytd+1])*); #Update the ytd of the wrhs
	dx(*$= (d\_ytd\_by\_id).\texttt{GET}$ (id=dist\_id)*) #Retrieve d_ytd by PK
	(*$ (d\_ytd\_by\_id).\texttt{PUT}$(dx[d\_ytd $\mapsto$d\_ytd+1])*); #Update the ytd of the district
	
	# Retrive customer info (except c_balance):
	IF ((*cust\_id*) = NULL) #Retrieve by id or name?
	THEN cx1(*$ = (c\_info\_by\_name).\texttt{GET}$ (c\_name=cust\_name)*);
	     cx = (*$\texttt{CHOOSE}\; cx1 $*) # pick the middle customer;
	ELSE cx(*$= (c\_info\_by\_id).\texttt{GET}$ (id=(cust\_id,...))*) #Retrieve customers by PK
	(*$ (c\_info\_by\_id).\texttt{PUT}$(cx*)
						(*[c\_ytd\_payment$\mapsto$c\_ytd\_payment+amnt]*)
						(*[c\_payment\_cnt$\mapsto$c\_payment\_cnt+1])*);
	
	# Retrive and update customer's balance:
	IF ((*cust\_id*) = NULL) #Retrieve by id or name?
	THEN cbx1(*$ = (c\_balance\_by\_name).\texttt{GET}$ (c\_name=cust\_name)*);
	     cbx = (*$\texttt{CHOOSE}\; cbx1 $*) # pick the middle customer;
	     #Update both customer objects:
	     (*$ (c\_balance\_by\_id).\texttt{PUT}$(cbx [c\_balance$\mapsto$c\_balance-amnt])*);
	     (*$ (c\_balance\_by\_name).\texttt{PUT}$(cbx [c\_balance$\mapsto$c\_balance-amnt])*);
	ELSE cbx(*$= (c\_balance\_by\_id).\texttt{GET}$ (id=(cust\_id,...))*)#Retrieve customers by PK
	     #Retrieve the same customer's info
	     cix(*$= (c\_info\_by\_id).\texttt{GET}$ (id=(cust\_id,...))*)#Retrieve customer by PK
			 # Update both objects:
	     (*$ (c\_balance\_by\_id).\texttt{PUT}$(cbx [c\_balance$\mapsto$c\_balance-amnt])*);
			 (*$ (c\_balance\_by\_name).\texttt{PUT}$((cix.name,cust\_id,cbx.c\_balance-amnt))*);
	(*$ (h\_info\_by\_id).\texttt{PUT}$(wh\_id,dist\_id,...)*);		 
}(*$_{SER}$*)
\end{lstlisting}


\hrule
\hrule
\hrule
\hrule
\hrule

%%%%%%%%%%%%%%%%%%%%%%%%%%%%%%%%%%%%%%%%%%%%%%%%%%%%%%%%%%%%%%%%%%%%%%%%%%%%%%%%%%%%%%%%%%%%%%%%%%%%%%5
%%%%%%%%%%%%%%%%%%%%%%%%%%%%%%%%%%%%%%%%%%%%%%%%%%%%%%%%%%%%%%%%%%%%%%%%%%%%%%%%%%%%%%%%%%%%%%%%%%%%%%5


\vspace{5mm}
\lstset{style=simpSQLstyle}
\paragraph{Order Status}:
%--------- Order Status
\begin{lstlisting}[language=Python, caption=simpSQL,escapeinside={(*}{*)}]
OrderStatus((*cust\_id,cust\_name*)) := 
	(*$ \;\texttt{IF}\; cust\_id = \texttt{NULL} $*) (*$\;\texttt{THEN}\; $*) 
		(*$\texttt{SELECT}\; (c\_id,c\_info,...) \;\texttt{AS}\; cx1 \;\texttt{WHERE}\; c\_name = cust\_name \wedge ...$*)
		(*$cx = \texttt{CHOOSE}\; cx1 $*) 
	(*$ \;\texttt{ELSE}\; $*) (*$\texttt{SELECT}\; (c\_id,c\_info,...) \;\texttt{AS}\; cx \;\texttt{WHERE}\; c\_id = cust\_id \wedge ...$*)
		(*$\texttt{SELECT}\; (o\_id,...) \;\texttt{AS}\; ox1 \;\texttt{WHERE}\; o\_c\_id = cx^1 \wedge o\_d\_id = dist\_id \wedge ...$ *)
		(*$ox = \texttt{CHOOSE}\; ox1 $*) 
	(*$ \;\texttt{SELECT}\; (ol\_info,...) \;\texttt{AS}\; olx \;\texttt{WHERE}\; ol\_o\_id = ox^1 \wedge ol\_d\_id = dist\_id \wedge ...$ *)
	print olx  
	
\end{lstlisting}

\lstset{style=kvSQLstyle}
%--------- Order Status
\begin{lstlisting}[language=Python, caption=kvSQL,escapeinside={(*}{*)}]
OrderStatus((*cust\_id,cust\_name*)) := {
	IF ((*cust\_id*) = NULL) #Retrieve by id or name?
	THEN cx1(*$ = (c\_info\_by\_name).\texttt{GET}$ (c\_name=cust\_name)*);
	     cx = (*$\texttt{CHOOSE}\; cx1 $*) # pick the middle customer;
	ELSE cx(*$= (c\_info\_by\_id).\texttt{GET}$ (id=(cust\_id,...))*) #Retrieve customers by PK
	ox1(*$ = (o\_info\_by\_o\_c\_id).\texttt{GET}$(o\_c\_id=cx.id)*); #Retrieve orders by non-PK
	ox =  (*$\texttt{CHOOSE}\;$ ox1 *);	# pick the largest order o_id
	olx(*$=(ol\_info\_by\_ol\_o\_id).\texttt{GET}$(ol\_o\_id=ox.o\_id)*); #Retrieve OrdLn by non-PK
	print olx  
}(*$_{SER}$*)
\end{lstlisting}


%%%%%%%%%%%%%%%%%%%%%%%%%%%%%%%%%%%%%%%%%%%%%%%%%%%%%%%%%%%%%%%%%%%%%%%%%%%%%%%%%%%%%%%%%%%%%%%%%%%%%%5
%%%%%%%%%%%%%%%%%%%%%%%%%%%%%%%%%%%%%%%%%%%%%%%%%%%%%%%%%%%%%%%%%%%%%%%%%%%%%%%%%%%%%%%%%%%%%%%%%%%%%%5

\hrule
\hrule
\hrule
\hrule
\hrule

\vspace{10mm}
\lstset{style=simpSQLstyle}
\paragraph{Stock Level}:
%--------- Stock Level
\begin{lstlisting}[language=Python, caption=simpSQL, escapeinside={(*}{*)}]
StockLevel((*dist\_id,wh\_id,thrshld*)) := 
	(*$ \;\texttt{SELECT}\; (d\_next\_o\_id) \;\texttt{AS}\; dnox \;\texttt{WHERE}\; d\_id=dist\_id \wedge  d\_w\_id = wh\_id$*)
	(*$ \;\texttt{SELECT}_{ol\_i\_id\bowtie s\_i\_id}\; (s\_info) \;\texttt{AS}\; sx$*)
		(*$\;\texttt{WHERE}\; (ol\_o\_id < dnox^1) \wedge (ol\_o\_id > dnox^1-20) \wedge (s\_quant < thrshld) \wedge ... $*)
	print sx
\end{lstlisting}

\lstset{style=kvSQLstyle}
\paragraph{Stock Level}:
%--------- Stock Level
\begin{lstlisting}[language=Python, caption=kvSQL, escapeinside={(*}{*)}]
StockLevel((*dist\_id,wh\_id*)) := {
	#Retrieve d_next_o_id by PK:
	dnox(*$=(d\_next\_o\_id\_by\_id).\texttt{GET}$ (id=(wh\_id,dist\_id))*)
	sqx1 (*$ = (s\_quant\_by\_ol\_o\_id).\texttt{GET}$(ol\_o\_id=dnox.next\_o\_id)*)
	sqx = (*$\texttt{FILTER}$ sqx1 *) #Filter by w_id and d_id and by s_quant
	print sqx 
}(*$_{SER}$*)
\end{lstlisting}


%%%%%%%%%%%%%%%%%%%%%%%%%%%%%%%%%%%%%%%%%%%%%%%%%%%%%%%%%%%%%%%%%%%%%%%%%%%%%%%%%%%%%%%%%%%%%%%%%%%%%%5
%%%%%%%%%%%%%%%%%%%%%%%%%%%%%%%%%%%%%%%%%%%%%%%%%%%%%%%%%%%%%%%%%%%%%%%%%%%%%%%%%%%%%%%%%%%%%%%%%%%%%%5

\hrule
\hrule
\hrule
\hrule
\hrule

\vspace{10 mm}
\lstset{style=simpSQLstyle}
\paragraph{Delivery}:
%--------- Delivery
\begin{lstlisting}[language=Python, caption=simpSQL,escapeinside={(*}{*)}]
Delivery(*(dist\_id,wh\_id, carr\_num, curr\_time)*) := 
	(*$ \;\texttt{SELECT}\; (no\_o\_id) \;\texttt{AS}\; nox1 \;\texttt{WHERE}\; no\_d\_id=dist\_id \wedge  no\_w\_id = wh\_id$*)
	(*$ \;nox = \texttt{CHOOSE}\;(nox1)$*)
	(*$ \;\texttt{DELETE}\; NewOrder \;\texttt{WHERE}\; (no\_o\_id=nox^1) \wedge (no\_d\_id=dist\_id) \wedge (no\_w\_id = wh\_id$)*)
	(*$ \;\texttt{SELECT}\; (o\_c\_id)  \;\texttt{AS}\; ocx \;\texttt{WHERE}\; (o\_id = nox^1) \wedge (o\_d\_id=dist\_id) \wedge...$*)
	(*$ \;\texttt{UPDATE SET}\; (o\_carrier\_id)=(carr\_num) \;\texttt{WHERE}\; (o\_id = nox^1) \wedge (o\_d\_id=dist\_id) \wedge...$*)
	(*$ \;\texttt{SELECT}\; ol\_amount \;\texttt{AS}\; olx \;\texttt{WHERE}\; ol\_o\_id=nox^1 \wedge ol\_d\_id=dist\_id \wedge ...$*)
	(*$ \;\texttt{UPDATE SET}\; ol\_delivery\_d=curr\_time \;\texttt{WHERE}\; ol\_o\_id=nox^1 \wedge ol\_d\_id=dist\_id \wedge ...$*)
	(*$ \;sumx = \texttt{CHOOSE}\;(olx)$*)
	(*$ \;\texttt{SELECT}\; (c\_balance,c\_deliveryCnt) \;\texttt{AS}\; cx \;\texttt{WHERE}\; o\_id = ocx^1 \wedge c\_d\_id=dist\_id \wedge...$*)
	(*$ \;\texttt{UPDATE SET}\; (c\_balance,c\_deliveryCnt)=(cx^1+sumx,cx^2+1) \;\texttt{WHERE}\; o\_id = ocx^1 \wedge...$*)
\end{lstlisting}


\lstset{style=kvSQLstyle}
%--------- Delivery
\begin{lstlisting}[language=Python, caption=kvSQL,escapeinside={(*}{*)}]
Delivery(*(dist\_id, carr\_num, curr\_time)*) := {
	nox2(*$ = (no\_by\_d\_id).\texttt{GET}$(no\_d\_id=dist\_id)*); #Retrive by partial key
	nox1 = (*$\texttt{FILTER}$*)(nox2); #Filter records by W_id 
	nox = (*$\texttt{CHOOSE}$*)(nox1); #Pick the record with the lowest o_id
	(*$(no\_by\_d\_id).\texttt{DELETE}$(id=(nox.o\_id,...))*); #Delete by PK 
	ox (*$ = (order\_by\_id).\texttt{GET}$(id=(nox.o\_id,...))*); #Retrive order by PK
	(*$ (order\_by\_id).\texttt{PUT}$(ox[o\_carier\_id $\mapsto$carr\_num])*); #Update the carrier ID
	(*$ (o\_id+...\_by\_o\_c\_id).\texttt{PUT}$(ox'[o\_carier\_id $\mapsto$carr\_num])*); #Update the carrier ID
	# ox' only includes interesting columns from ox
	olx1(*$=(ol\_info\_by\_ol\_o\_id).\texttt{GET}$(ol\_o\_id=nox.o\_id)*);
	olx = (*$\texttt{FILTER}\;$ olx1 *); #Filter by w_id and d_id 
	s = 0;
	FOREACH olr IN olx DO
		(*$ (ol\_info\_by\_ol\_o\_id).\texttt{PUT}$(olr[ol\_info$\mapsto$curr\_time])*);
		(*$ (ol\_info\_by\_id).\texttt{PUT}$(olr[ol\_info$\mapsto$curr\_time])*);
		(*$ s = s +$ olr.ol\_info *)
	END;
	cx(*$ \leftarrow (c\_info\_by\_id).\texttt{GET}$ (id=ox.c\_id)*) #Retrive customer by PK
	#Update c_info_by_id and c_balance_by_id:
	(*$ (c\_info\_by\_id).\texttt{PUT}$(cx[c\_delivery\_cnt $\mapsto$ c\_delivery\_cnt + 1])*);#update deliveryCnt
	(*$ (c\_info\_by\_id).\texttt{PUT}$(cx[c\_balance $\mapsto$ c\_balance - s])*); #update delivery cnt
	#Update c_info_by_name and c_balance_by_name:
	(*$ (c\_info\_by\_name).\texttt{PUT}$(cx'[c\_delvry\_cnt$\mapsto$c\_delvry\_cnt+1])*);#update deliveryCnt
	(*$ (c\_info\_by\_name).\texttt{PUT}$(cx'[c\_balance $\mapsto$ c\_balance - s])*); #update delivery cnt
}(*$_{SER}$*)
\end{lstlisting}



\subsection{Soundness of the Translation}
\textcolor{red}{//TODO}


% ---------------------------------------------------
\section{Optimization}
We create a more optimized version of the initial kvSQL program, either by
incrementally strengthening the weakest isolation level
(using Kartik's analysis and program patches) or by weakening the SER version
(how?).
\textcolor{red}{//TODO}


\subsection{Example: Optimized TPC-C}
\textcolor{red}{//TODO}

\subsection{Soundness of the Optimizer}
\textcolor{red}{//TODO}



% The Biblography
\bibliographystyle{}
\bibliography{../kia-bib}
\end{document}
