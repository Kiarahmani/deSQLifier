\title{\LaTeX Template}
\author{
        Kia Rahmani \\
                Department of Computer Science\\
        Purdue University, USA
}
\date{\today}

% Main Document
\documentclass[12pt,letter]{article}
\usepackage{xcolor,listings}
\usepackage{graphicx}
\usepackage[inline]{enumitem}
\usepackage[utf8]{inputenc}
\usepackage[english,ngerman]{babel}
\usepackage{amsmath}
\usepackage{mathtools}
\usepackage{mathpartir}
\usepackage{listings}
\usepackage{color}

\definecolor{codegreen}{rgb}{0,0.6,0}
\definecolor{codegray}{rgb}{0.5,0.5,0.5}
\definecolor{codepurple}{rgb}{0.58,0,0.82}
\definecolor{backcolour}{rgb}{0.95,0.95,0.92}
 
\lstdefinestyle{mystyle}{
    backgroundcolor=\color{backcolour},   
    commentstyle=\color{codegreen},
    keywordstyle=\color{magenta},
    numberstyle=\tiny\color{codegray},
    stringstyle=\color{codepurple},
    basicstyle=\footnotesize,
    breakatwhitespace=false,         
    breaklines=true,                 
    captionpos=b,                    
    keepspaces=true,                 
    numbers=left,                    
    numbersep=5pt,                  
    showspaces=false,                
    showstringspaces=false,
    showtabs=false,                  
    tabsize=2
}
 
\lstset{style=mystyle}



\begin{document}
\selectlanguage{english}
%\maketitle
%\begin{abstract} \end{abstract}


% Sections
% ---------------------------------------------------
\section{Syntax of simpSQL}
The following is the formal definition of the simpSQL language based on Kartik's document,
representing a simple programing language with realistic standard SQL queries.

% the syntax of standard SQL
\begin{figure}[h]
	$$
	t \in \texttt{TableName} \qquad 
	f_{id},f_v \in \texttt{FieldName} \qquad 
	v \in \texttt{Value} $$
	%
	\vspace{-7mm} %visually better
	%
	$$\qquad
	x \in \texttt{Variable} \qquad
	txn \in \texttt{TxnName}
	$$
	%
	\vspace{-6mm} %visually better
	%
	$$ \odot \in \{<,\leq,=,\neq,>,\geq\} \qquad 
	\oplus \in \{\cap,\cup\} \qquad 
	\otimes \in \{\wedge, \vee\}
	$$ 
	%
	$$
	\begin{matrix*}[l]
		pk & ::= & (f_{id},f_v)\\
		obj &  ::= & (t,pk,\overline{f_v}) \\
		r_{obj} &  ::= & \bar{v} \\
		\phi_{pk}  & ::= & pk_{id} \odot v \;|\; pk_{v} \odot v
		\;|\; \phi_{pk} \otimes \phi_{pk} \\
		e  & ::= & x \;|\; \texttt{CHOOSE} \; x\;|\; r_{obj} \;|\; e \oplus e\\
		\phi_{c}  & ::= & r^i_{obj}\odot v \;|\;  r\;\texttt{IN} \; e \;|\; \phi_{c} \otimes \phi_{c} \\
		op   & ::= & obj.put(r) \quad
		|\quad x\leftarrow obj.get(\phi_{obj})  \\
		c   & ::=  & \{\overline{op}\}_{DC}
		\;|\; x\leftarrow e \;|\;
		\\  & & \texttt{IF}\; \phi_c \;\texttt{THEN} \;c \;\texttt{ELSE}\; c \;|\;  c;c \;|\;
		\{c\}_{_{SER}} \;|\;
		\\  & & \texttt{FOREACH}\; r \;\texttt{IN} \; x \; \texttt{DO}\; c \;\texttt{END}
		
	\end{matrix*}
	$$
	\hrule \hrule 
\caption{Syntax of simpSQL}
\label{fig:standard}
\end{figure}



% ---------------------------------------------------
\section{Syntax of kvSQL}
Figure \ref{fig:syn} presents the kvSQL language which is used to write
generic key-value backed applications. The language is not very
different from SQL; it simply replaces tables with (denormalized)
objects supporting restricted queries. We will later formally define the
translation standard SQL to kvSQL\footnote{proving this translation
correct is NOT giong to be challenging, since we will initially translate the simpSQL
program to a kvSQL version \emph{with SER transactions everywhere} and the difference will
only be in the data models}.


% the syntax of kvSQL
\begin{figure}[h]
	$$
	t \in \texttt{TableName} \qquad 
	f_{id},f_v \in \texttt{FieldName} \qquad 
	v \in \texttt{Value} $$
	%
	\vspace{-7mm} %visually better
	%
	$$\qquad
	x \in \texttt{Variable} \qquad
	txn \in \texttt{TxnName}
	$$
	%
	\vspace{-6mm} %visually better
	%
	$$ \odot \in \{<,\leq,=,\neq,>,\geq\} \qquad 
	\oplus \in \{\cap,\cup\} \qquad 
	\otimes \in \{\wedge, \vee\}
	$$ 
	%
	$$
	\begin{matrix*}[l]
		pk & ::= & (f_{id},f_v)\\
		obj &  ::= & (t,pk,\overline{f_v}) \\
		r_{obj} &  ::= & \bar{v} \\
		\phi_{pk}  & ::= & pk_{id} \odot v \;|\; pk_{v} \odot v
		\;|\; \phi_{pk} \otimes \phi_{pk} \\
		e  & ::= & x \;|\; \texttt{CHOOSE} \; x\;|\; r_{obj} \;|\; e \oplus e\\
		\phi_{c}  & ::= & r^i_{obj}\odot v \;|\;  r\;\texttt{IN} \; e \;|\; \phi_{c} \otimes \phi_{c} \\
		op   & ::= & obj.put(r) \quad
		|\quad x\leftarrow obj.get(\phi_{obj})  \\
		c   & ::=  & \{\overline{op}\}_{DC}
		\;|\; x\leftarrow e \;|\;
		\\  & & \texttt{IF}\; \phi_c \;\texttt{THEN} \;c \;\texttt{ELSE}\; c \;|\;  c;c \;|\;
		\{c\}_{_{SER}} \;|\;
		\\  & & \texttt{FOREACH}\; r \;\texttt{IN} \; x \; \texttt{DO}\; c \;\texttt{END}
		
	\end{matrix*}
	$$
	\hrule \hrule 
\caption{Syntax of kvSQL}
\label{fig:syn}
\end{figure}


% ---------------------------------------------------
\section{Definition of the denormalizer}
\subsection{Data Modeling Rules}

\subsection{Program rewriting rules}




% ---------------------------------------------------
\newpage
\section{Example: TPC-C in simpSQL and kvSQL}

% WAREHOUSE
\subsubsection*{SimpSQL Table: Warehouse}  
\begin{tabular}{ |c|c|c|c|c| }
 \hline
 \underline{w\_id} & w\_name & w\_address & w\_tax & w\_ytd \\
 \hline
 &   &   & & \\
 \hline
\end{tabular}

\subsubsection*{kvSQL Object(s): Warehouse}  
id := (w\_id) \\
warehouse\_by\_id :=
(Warehouse,(id,\_),[w\_name;w\_address;w\_tax;w\_ytd]) 
\\ \\
\hrule

%DISTRICT 
\subsubsection*{SimpSQL Table: District}  
\begin{tabular}{ |c|c|c|c|c|c| }
 \hline
 \underline{d\_id} & \underline{d\_w\_id} & d\_info & d\_ytd & d\_tax & d\_next\_o\_id\\
 \hline
 &   &   & & &\\
 \hline
\end{tabular}

\subsubsection*{kvSQL Object(s): District}  
 id := (d\_id,d\_w\_id) \\
 d\_info\_by\_id := 
(District,(id,\_),[d\_info])  \\
 d\_ytd\_by\_id := 
(District,(id,\_),[d\_ytd])  \\
 d\_tax\_by\_id := 
(District,(id,\_),[d\_tax])  \\
 d\_next\_o\_id\_by\_id := 
(District,(id,\_),[d\_next\_o\_id])  \\
\\ 
\hrule

%CUSTOMER
\subsubsection*{SimpSQL Table: Customer}  
\begin{tabular}{ |c|c|c|c|c|c|c|c| }
 \hline
 \underline{c\_id} & \underline{c\_d\_id} & \underline{c\_w\_id} &
 c\_name & c\_ytd &
 c\_delivery\_cnt & c\_payment\_cnt & c\_balance\\
 \hline
 &   &   & & & & &\\
 \hline
\end{tabular}

\subsubsection*{kvSQL Object(s): Customer}  
id := (c\_id,c\_d\_id,c\_w\_id) \\
 c\_name+ytd+...\_by\_id := 
(Customer,(id,\_),[c\_name;c\_ytd;...])  \\
 c\_balance\_by\_id := 
(Customer,(id,\_),[c\_balance])  \\
 c\_ytd+...\_by\_name := 
(Customer,(id,c\_name),[c\_ytd;...])  \\
 c\_balance\_by\_name := 
(Customer,(id,c\_name),[c\_balance])  \\
\\

%ORDERS
\subsubsection*{SimpSQL Table: Orders}  
\begin{tabular}{ |c|c|c|c|c|c| }
 \hline
 \underline{o\_id} & \underline{o\_d\_id} & \underline{o\_w\_id} &
 o\_c\_id & o\_carrier\_id & o\_entry\_d\\
 \hline
 &   &   & & &\\
 \hline
\end{tabular}

\subsubsection*{kvSQL Object(s): Orders}  
id := (o\_id,o\_d\_id,o\_w\_id) \\
order\_by\_id := 
(Orders,(id,\_),[o\_c\_id;o\_carrier\_id;o\_entry\_d])  \\
o\_id+entryD+CarriedID\_by\_o\_c\_id := 
(Orders,(id,o\_c\_id),[o\_id;...])  \\
\\ 
\hrule


%ITEM
\subsubsection*{SimpSQL Table: Item}  
\begin{tabular}{ |c|c| }
 \hline
 \underline{i\_id} & i\_info\\
 \hline
 &   \\
 \hline
\end{tabular}

\subsubsection*{kvSQL Object(s): Item}  
id := (i\_id)\\
 i\_info\_by\_id := 
(Item,(id,\_),[i\_info]) \\
\\
\hrule


% ORDERLINE
\subsubsection*{SimpSQL Table: OrderLine}  
\begin{tabular}{ |c|c|c|c|c| }
 \hline
 \underline{ol\_o\_id} & \underline{ol\_d\_id} & \underline{ol\_w\_id} &
 \underline{ol\_number} & ol\_info\\
 \hline
 &   &   & & \\
 \hline
\end{tabular}

\subsubsection*{kvSQL Object(s): OrderLine}  
id := (ol\_o\_id,ol\_d\_id,ol\_w\_id,ol\_number) \\
ol\_info\_by\_id :=
(OrderLine,(id,\_),[ol\_info]) \\
ol\_number+info\_by\_ol\_o\_id :=
(OrderLine,(id,ol\_o\_id),[ol\_number;ol\_info]) \\
\\ 
\hrule


% STOCK
\subsubsection*{SimpSQL Table: Stock}  
\begin{tabular}{ |c|c|c|c|c| }
 \hline
 \underline{s\_i\_id} & \underline{s\_w\_id} & s\_quant & s\_order\_cnt &
 s\_info \\
 \hline
 &   &   & & \\
 \hline
\end{tabular}

\subsubsection*{kvSQL Object(s): Stock}  
id := (s\_i\_id,s\_w\_id) \\
s\_quant\_by\_id :=
(Stock,(id,\_),[s\_quant]) \\
s\_orderCnt\_by\_id :=
(Stock,(id,\_),[s\_order\_cnt]) \\
s\_info\_by\_id :=
(Stock,(id,\_),[s\_info]) \\
\\ 
\hrule


% ORDERLINE |><| STOCK 
\subsubsection*{SimpSQL Table: OrderLine JOIN Stock}
\begin{tabular}{ |c|c|c|c|c|c|c|c| }
 \hline
 \underline{ol\_o\_id} & \underline{ol\_d\_id} & \underline{ol\_w\_id} &
 \underline{ol\_number} & ol\_info &s\_i\_id & s\_w\_id & s\_quant \\
 \hline
 &   &   & & & & &\\
 \hline
\end{tabular}

\subsubsection*{kvSQL Object(s): OrderLine JOIN Stock}  
id := (ol\_o\_id,ol\_d\_id,ol\_w\_id,ol\_number) \\
s\_quant\_by\_ol\_o\_id :=
(OrderLine $\bowtie$ Stock,(id,ol\_o\_id),[s\_quant]) \\
\\ 
\hrule


% NEWORDER
\subsubsection*{SimpSQL Table: NewOrder}
\begin{tabular}{ |c|c|c| }
 \hline
 \underline{ol\_o\_id} & \underline{ol\_d\_id} & \underline{ol\_w\_id}\\
 \hline
 &   &  \\
 \hline
\end{tabular}

\subsubsection*{kvSQL Object(s): NewOrder}  
id := (no\_o\_id,no\_d\_id,no\_w\_id) \\
?\_by\_no\_d\_id :=
(NewOrder,(id,no\_d\_id),[]) \\
\\ 
\hrule

%History
\subsubsection*{SimpSQL Table: History}  
\begin{tabular}{ |c|c| }
 \hline
 \underline{h\_id} & h\_info\\
 \hline
 &   \\
 \hline
\end{tabular}

\subsubsection*{kvSQL Object(s): History}  
id := (h\_id)\\
 h\_info\_by\_id := 
(Item,(id,\_),[h\_info]) \\
\\
\hrule










\newpage
\subsection*{simpSQL TPC-C}







\subsection*{kvSQL TPC-C}


\subsubsection{kv transactions}
%--------- New Order
\begin{lstlisting}[language=Python, caption=NewOrder Transaction]
new order
NEW ORDER!
\end{lstlisting}

%--------- Payment
\begin{lstlisting}[language=Python, caption=Payment Transaction]
payment
payment
payment
payment
payment
\end{lstlisting}


%--------- Order Status
\begin{lstlisting}[language=Python, caption=OrderStatus Transaction]
order status
order status
order status
order status
\end{lstlisting}


%--------- Stock Level
\begin{lstlisting}[language=Python, caption=StockLevel Transaction]
Stock level
Stock level
Stock level
Stock level
Stock level
\end{lstlisting}


%--------- Delivery
\begin{lstlisting}[language=Python, caption=Delivery Transaction]
delivery
delivery
delivery
delivery
delivery
\end{lstlisting}







% The Biblography
\bibliographystyle{}
\bibliography{../kia-bib}
\end{document}
